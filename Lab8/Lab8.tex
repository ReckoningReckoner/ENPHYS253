
% Default to the notebook output style

    


% Inherit from the specified cell style.




    
\documentclass[11pt]{article}

    
    
    \usepackage[T1]{fontenc}
    % Nicer default font (+ math font) than Computer Modern for most use cases
    \usepackage{mathpazo}

    % Basic figure setup, for now with no caption control since it's done
    % automatically by Pandoc (which extracts ![](path) syntax from Markdown).
    \usepackage{graphicx}
    % We will generate all images so they have a width \maxwidth. This means
    % that they will get their normal width if they fit onto the page, but
    % are scaled down if they would overflow the margins.
    \makeatletter
    \def\maxwidth{\ifdim\Gin@nat@width>\linewidth\linewidth
    \else\Gin@nat@width\fi}
    \makeatother
    \let\Oldincludegraphics\includegraphics
    % Set max figure width to be 80% of text width, for now hardcoded.
    \renewcommand{\includegraphics}[1]{\Oldincludegraphics[width=.8\maxwidth]{#1}}
    % Ensure that by default, figures have no caption (until we provide a
    % proper Figure object with a Caption API and a way to capture that
    % in the conversion process - todo).
    \usepackage{caption}
    \DeclareCaptionLabelFormat{nolabel}{}
    \captionsetup{labelformat=nolabel}

    \usepackage{adjustbox} % Used to constrain images to a maximum size 
    \usepackage{xcolor} % Allow colors to be defined
    \usepackage{enumerate} % Needed for markdown enumerations to work
    \usepackage{geometry} % Used to adjust the document margins
    \usepackage{amsmath} % Equations
    \usepackage{amssymb} % Equations
    \usepackage{textcomp} % defines textquotesingle
    % Hack from http://tex.stackexchange.com/a/47451/13684:
    \AtBeginDocument{%
        \def\PYZsq{\textquotesingle}% Upright quotes in Pygmentized code
    }
    \usepackage{upquote} % Upright quotes for verbatim code
    \usepackage{eurosym} % defines \euro
    \usepackage[mathletters]{ucs} % Extended unicode (utf-8) support
    \usepackage[utf8x]{inputenc} % Allow utf-8 characters in the tex document
    \usepackage{fancyvrb} % verbatim replacement that allows latex
    \usepackage{grffile} % extends the file name processing of package graphics 
                         % to support a larger range 
    % The hyperref package gives us a pdf with properly built
    % internal navigation ('pdf bookmarks' for the table of contents,
    % internal cross-reference links, web links for URLs, etc.)
    \usepackage{hyperref}
    \usepackage{longtable} % longtable support required by pandoc >1.10
    \usepackage{booktabs}  % table support for pandoc > 1.12.2
    \usepackage[inline]{enumitem} % IRkernel/repr support (it uses the enumerate* environment)
    \usepackage[normalem]{ulem} % ulem is needed to support strikethroughs (\sout)
                                % normalem makes italics be italics, not underlines
    

    
    
    % Colors for the hyperref package
    \definecolor{urlcolor}{rgb}{0,.145,.698}
    \definecolor{linkcolor}{rgb}{.71,0.21,0.01}
    \definecolor{citecolor}{rgb}{.12,.54,.11}

    % ANSI colors
    \definecolor{ansi-black}{HTML}{3E424D}
    \definecolor{ansi-black-intense}{HTML}{282C36}
    \definecolor{ansi-red}{HTML}{E75C58}
    \definecolor{ansi-red-intense}{HTML}{B22B31}
    \definecolor{ansi-green}{HTML}{00A250}
    \definecolor{ansi-green-intense}{HTML}{007427}
    \definecolor{ansi-yellow}{HTML}{DDB62B}
    \definecolor{ansi-yellow-intense}{HTML}{B27D12}
    \definecolor{ansi-blue}{HTML}{208FFB}
    \definecolor{ansi-blue-intense}{HTML}{0065CA}
    \definecolor{ansi-magenta}{HTML}{D160C4}
    \definecolor{ansi-magenta-intense}{HTML}{A03196}
    \definecolor{ansi-cyan}{HTML}{60C6C8}
    \definecolor{ansi-cyan-intense}{HTML}{258F8F}
    \definecolor{ansi-white}{HTML}{C5C1B4}
    \definecolor{ansi-white-intense}{HTML}{A1A6B2}

    % commands and environments needed by pandoc snippets
    % extracted from the output of `pandoc -s`
    \providecommand{\tightlist}{%
      \setlength{\itemsep}{0pt}\setlength{\parskip}{0pt}}
    \DefineVerbatimEnvironment{Highlighting}{Verbatim}{commandchars=\\\{\}}
    % Add ',fontsize=\small' for more characters per line
    \newenvironment{Shaded}{}{}
    \newcommand{\KeywordTok}[1]{\textcolor[rgb]{0.00,0.44,0.13}{\textbf{{#1}}}}
    \newcommand{\DataTypeTok}[1]{\textcolor[rgb]{0.56,0.13,0.00}{{#1}}}
    \newcommand{\DecValTok}[1]{\textcolor[rgb]{0.25,0.63,0.44}{{#1}}}
    \newcommand{\BaseNTok}[1]{\textcolor[rgb]{0.25,0.63,0.44}{{#1}}}
    \newcommand{\FloatTok}[1]{\textcolor[rgb]{0.25,0.63,0.44}{{#1}}}
    \newcommand{\CharTok}[1]{\textcolor[rgb]{0.25,0.44,0.63}{{#1}}}
    \newcommand{\StringTok}[1]{\textcolor[rgb]{0.25,0.44,0.63}{{#1}}}
    \newcommand{\CommentTok}[1]{\textcolor[rgb]{0.38,0.63,0.69}{\textit{{#1}}}}
    \newcommand{\OtherTok}[1]{\textcolor[rgb]{0.00,0.44,0.13}{{#1}}}
    \newcommand{\AlertTok}[1]{\textcolor[rgb]{1.00,0.00,0.00}{\textbf{{#1}}}}
    \newcommand{\FunctionTok}[1]{\textcolor[rgb]{0.02,0.16,0.49}{{#1}}}
    \newcommand{\RegionMarkerTok}[1]{{#1}}
    \newcommand{\ErrorTok}[1]{\textcolor[rgb]{1.00,0.00,0.00}{\textbf{{#1}}}}
    \newcommand{\NormalTok}[1]{{#1}}
    
    % Additional commands for more recent versions of Pandoc
    \newcommand{\ConstantTok}[1]{\textcolor[rgb]{0.53,0.00,0.00}{{#1}}}
    \newcommand{\SpecialCharTok}[1]{\textcolor[rgb]{0.25,0.44,0.63}{{#1}}}
    \newcommand{\VerbatimStringTok}[1]{\textcolor[rgb]{0.25,0.44,0.63}{{#1}}}
    \newcommand{\SpecialStringTok}[1]{\textcolor[rgb]{0.73,0.40,0.53}{{#1}}}
    \newcommand{\ImportTok}[1]{{#1}}
    \newcommand{\DocumentationTok}[1]{\textcolor[rgb]{0.73,0.13,0.13}{\textit{{#1}}}}
    \newcommand{\AnnotationTok}[1]{\textcolor[rgb]{0.38,0.63,0.69}{\textbf{\textit{{#1}}}}}
    \newcommand{\CommentVarTok}[1]{\textcolor[rgb]{0.38,0.63,0.69}{\textbf{\textit{{#1}}}}}
    \newcommand{\VariableTok}[1]{\textcolor[rgb]{0.10,0.09,0.49}{{#1}}}
    \newcommand{\ControlFlowTok}[1]{\textcolor[rgb]{0.00,0.44,0.13}{\textbf{{#1}}}}
    \newcommand{\OperatorTok}[1]{\textcolor[rgb]{0.40,0.40,0.40}{{#1}}}
    \newcommand{\BuiltInTok}[1]{{#1}}
    \newcommand{\ExtensionTok}[1]{{#1}}
    \newcommand{\PreprocessorTok}[1]{\textcolor[rgb]{0.74,0.48,0.00}{{#1}}}
    \newcommand{\AttributeTok}[1]{\textcolor[rgb]{0.49,0.56,0.16}{{#1}}}
    \newcommand{\InformationTok}[1]{\textcolor[rgb]{0.38,0.63,0.69}{\textbf{\textit{{#1}}}}}
    \newcommand{\WarningTok}[1]{\textcolor[rgb]{0.38,0.63,0.69}{\textbf{\textit{{#1}}}}}
    
    
    % Define a nice break command that doesn't care if a line doesn't already
    % exist.
    \def\br{\hspace*{\fill} \\* }
    % Math Jax compatability definitions
    \def\gt{>}
    \def\lt{<}
    % Document parameters
    \title{Lab8}
    
    
    

    % Pygments definitions
    
\makeatletter
\def\PY@reset{\let\PY@it=\relax \let\PY@bf=\relax%
    \let\PY@ul=\relax \let\PY@tc=\relax%
    \let\PY@bc=\relax \let\PY@ff=\relax}
\def\PY@tok#1{\csname PY@tok@#1\endcsname}
\def\PY@toks#1+{\ifx\relax#1\empty\else%
    \PY@tok{#1}\expandafter\PY@toks\fi}
\def\PY@do#1{\PY@bc{\PY@tc{\PY@ul{%
    \PY@it{\PY@bf{\PY@ff{#1}}}}}}}
\def\PY#1#2{\PY@reset\PY@toks#1+\relax+\PY@do{#2}}

\expandafter\def\csname PY@tok@w\endcsname{\def\PY@tc##1{\textcolor[rgb]{0.73,0.73,0.73}{##1}}}
\expandafter\def\csname PY@tok@c\endcsname{\let\PY@it=\textit\def\PY@tc##1{\textcolor[rgb]{0.25,0.50,0.50}{##1}}}
\expandafter\def\csname PY@tok@cp\endcsname{\def\PY@tc##1{\textcolor[rgb]{0.74,0.48,0.00}{##1}}}
\expandafter\def\csname PY@tok@k\endcsname{\let\PY@bf=\textbf\def\PY@tc##1{\textcolor[rgb]{0.00,0.50,0.00}{##1}}}
\expandafter\def\csname PY@tok@kp\endcsname{\def\PY@tc##1{\textcolor[rgb]{0.00,0.50,0.00}{##1}}}
\expandafter\def\csname PY@tok@kt\endcsname{\def\PY@tc##1{\textcolor[rgb]{0.69,0.00,0.25}{##1}}}
\expandafter\def\csname PY@tok@o\endcsname{\def\PY@tc##1{\textcolor[rgb]{0.40,0.40,0.40}{##1}}}
\expandafter\def\csname PY@tok@ow\endcsname{\let\PY@bf=\textbf\def\PY@tc##1{\textcolor[rgb]{0.67,0.13,1.00}{##1}}}
\expandafter\def\csname PY@tok@nb\endcsname{\def\PY@tc##1{\textcolor[rgb]{0.00,0.50,0.00}{##1}}}
\expandafter\def\csname PY@tok@nf\endcsname{\def\PY@tc##1{\textcolor[rgb]{0.00,0.00,1.00}{##1}}}
\expandafter\def\csname PY@tok@nc\endcsname{\let\PY@bf=\textbf\def\PY@tc##1{\textcolor[rgb]{0.00,0.00,1.00}{##1}}}
\expandafter\def\csname PY@tok@nn\endcsname{\let\PY@bf=\textbf\def\PY@tc##1{\textcolor[rgb]{0.00,0.00,1.00}{##1}}}
\expandafter\def\csname PY@tok@ne\endcsname{\let\PY@bf=\textbf\def\PY@tc##1{\textcolor[rgb]{0.82,0.25,0.23}{##1}}}
\expandafter\def\csname PY@tok@nv\endcsname{\def\PY@tc##1{\textcolor[rgb]{0.10,0.09,0.49}{##1}}}
\expandafter\def\csname PY@tok@no\endcsname{\def\PY@tc##1{\textcolor[rgb]{0.53,0.00,0.00}{##1}}}
\expandafter\def\csname PY@tok@nl\endcsname{\def\PY@tc##1{\textcolor[rgb]{0.63,0.63,0.00}{##1}}}
\expandafter\def\csname PY@tok@ni\endcsname{\let\PY@bf=\textbf\def\PY@tc##1{\textcolor[rgb]{0.60,0.60,0.60}{##1}}}
\expandafter\def\csname PY@tok@na\endcsname{\def\PY@tc##1{\textcolor[rgb]{0.49,0.56,0.16}{##1}}}
\expandafter\def\csname PY@tok@nt\endcsname{\let\PY@bf=\textbf\def\PY@tc##1{\textcolor[rgb]{0.00,0.50,0.00}{##1}}}
\expandafter\def\csname PY@tok@nd\endcsname{\def\PY@tc##1{\textcolor[rgb]{0.67,0.13,1.00}{##1}}}
\expandafter\def\csname PY@tok@s\endcsname{\def\PY@tc##1{\textcolor[rgb]{0.73,0.13,0.13}{##1}}}
\expandafter\def\csname PY@tok@sd\endcsname{\let\PY@it=\textit\def\PY@tc##1{\textcolor[rgb]{0.73,0.13,0.13}{##1}}}
\expandafter\def\csname PY@tok@si\endcsname{\let\PY@bf=\textbf\def\PY@tc##1{\textcolor[rgb]{0.73,0.40,0.53}{##1}}}
\expandafter\def\csname PY@tok@se\endcsname{\let\PY@bf=\textbf\def\PY@tc##1{\textcolor[rgb]{0.73,0.40,0.13}{##1}}}
\expandafter\def\csname PY@tok@sr\endcsname{\def\PY@tc##1{\textcolor[rgb]{0.73,0.40,0.53}{##1}}}
\expandafter\def\csname PY@tok@ss\endcsname{\def\PY@tc##1{\textcolor[rgb]{0.10,0.09,0.49}{##1}}}
\expandafter\def\csname PY@tok@sx\endcsname{\def\PY@tc##1{\textcolor[rgb]{0.00,0.50,0.00}{##1}}}
\expandafter\def\csname PY@tok@m\endcsname{\def\PY@tc##1{\textcolor[rgb]{0.40,0.40,0.40}{##1}}}
\expandafter\def\csname PY@tok@gh\endcsname{\let\PY@bf=\textbf\def\PY@tc##1{\textcolor[rgb]{0.00,0.00,0.50}{##1}}}
\expandafter\def\csname PY@tok@gu\endcsname{\let\PY@bf=\textbf\def\PY@tc##1{\textcolor[rgb]{0.50,0.00,0.50}{##1}}}
\expandafter\def\csname PY@tok@gd\endcsname{\def\PY@tc##1{\textcolor[rgb]{0.63,0.00,0.00}{##1}}}
\expandafter\def\csname PY@tok@gi\endcsname{\def\PY@tc##1{\textcolor[rgb]{0.00,0.63,0.00}{##1}}}
\expandafter\def\csname PY@tok@gr\endcsname{\def\PY@tc##1{\textcolor[rgb]{1.00,0.00,0.00}{##1}}}
\expandafter\def\csname PY@tok@ge\endcsname{\let\PY@it=\textit}
\expandafter\def\csname PY@tok@gs\endcsname{\let\PY@bf=\textbf}
\expandafter\def\csname PY@tok@gp\endcsname{\let\PY@bf=\textbf\def\PY@tc##1{\textcolor[rgb]{0.00,0.00,0.50}{##1}}}
\expandafter\def\csname PY@tok@go\endcsname{\def\PY@tc##1{\textcolor[rgb]{0.53,0.53,0.53}{##1}}}
\expandafter\def\csname PY@tok@gt\endcsname{\def\PY@tc##1{\textcolor[rgb]{0.00,0.27,0.87}{##1}}}
\expandafter\def\csname PY@tok@err\endcsname{\def\PY@bc##1{\setlength{\fboxsep}{0pt}\fcolorbox[rgb]{1.00,0.00,0.00}{1,1,1}{\strut ##1}}}
\expandafter\def\csname PY@tok@kc\endcsname{\let\PY@bf=\textbf\def\PY@tc##1{\textcolor[rgb]{0.00,0.50,0.00}{##1}}}
\expandafter\def\csname PY@tok@kd\endcsname{\let\PY@bf=\textbf\def\PY@tc##1{\textcolor[rgb]{0.00,0.50,0.00}{##1}}}
\expandafter\def\csname PY@tok@kn\endcsname{\let\PY@bf=\textbf\def\PY@tc##1{\textcolor[rgb]{0.00,0.50,0.00}{##1}}}
\expandafter\def\csname PY@tok@kr\endcsname{\let\PY@bf=\textbf\def\PY@tc##1{\textcolor[rgb]{0.00,0.50,0.00}{##1}}}
\expandafter\def\csname PY@tok@bp\endcsname{\def\PY@tc##1{\textcolor[rgb]{0.00,0.50,0.00}{##1}}}
\expandafter\def\csname PY@tok@vc\endcsname{\def\PY@tc##1{\textcolor[rgb]{0.10,0.09,0.49}{##1}}}
\expandafter\def\csname PY@tok@vg\endcsname{\def\PY@tc##1{\textcolor[rgb]{0.10,0.09,0.49}{##1}}}
\expandafter\def\csname PY@tok@vi\endcsname{\def\PY@tc##1{\textcolor[rgb]{0.10,0.09,0.49}{##1}}}
\expandafter\def\csname PY@tok@sb\endcsname{\def\PY@tc##1{\textcolor[rgb]{0.73,0.13,0.13}{##1}}}
\expandafter\def\csname PY@tok@sc\endcsname{\def\PY@tc##1{\textcolor[rgb]{0.73,0.13,0.13}{##1}}}
\expandafter\def\csname PY@tok@s2\endcsname{\def\PY@tc##1{\textcolor[rgb]{0.73,0.13,0.13}{##1}}}
\expandafter\def\csname PY@tok@sh\endcsname{\def\PY@tc##1{\textcolor[rgb]{0.73,0.13,0.13}{##1}}}
\expandafter\def\csname PY@tok@s1\endcsname{\def\PY@tc##1{\textcolor[rgb]{0.73,0.13,0.13}{##1}}}
\expandafter\def\csname PY@tok@mb\endcsname{\def\PY@tc##1{\textcolor[rgb]{0.40,0.40,0.40}{##1}}}
\expandafter\def\csname PY@tok@mf\endcsname{\def\PY@tc##1{\textcolor[rgb]{0.40,0.40,0.40}{##1}}}
\expandafter\def\csname PY@tok@mh\endcsname{\def\PY@tc##1{\textcolor[rgb]{0.40,0.40,0.40}{##1}}}
\expandafter\def\csname PY@tok@mi\endcsname{\def\PY@tc##1{\textcolor[rgb]{0.40,0.40,0.40}{##1}}}
\expandafter\def\csname PY@tok@il\endcsname{\def\PY@tc##1{\textcolor[rgb]{0.40,0.40,0.40}{##1}}}
\expandafter\def\csname PY@tok@mo\endcsname{\def\PY@tc##1{\textcolor[rgb]{0.40,0.40,0.40}{##1}}}
\expandafter\def\csname PY@tok@ch\endcsname{\let\PY@it=\textit\def\PY@tc##1{\textcolor[rgb]{0.25,0.50,0.50}{##1}}}
\expandafter\def\csname PY@tok@cm\endcsname{\let\PY@it=\textit\def\PY@tc##1{\textcolor[rgb]{0.25,0.50,0.50}{##1}}}
\expandafter\def\csname PY@tok@cpf\endcsname{\let\PY@it=\textit\def\PY@tc##1{\textcolor[rgb]{0.25,0.50,0.50}{##1}}}
\expandafter\def\csname PY@tok@c1\endcsname{\let\PY@it=\textit\def\PY@tc##1{\textcolor[rgb]{0.25,0.50,0.50}{##1}}}
\expandafter\def\csname PY@tok@cs\endcsname{\let\PY@it=\textit\def\PY@tc##1{\textcolor[rgb]{0.25,0.50,0.50}{##1}}}

\def\PYZbs{\char`\\}
\def\PYZus{\char`\_}
\def\PYZob{\char`\{}
\def\PYZcb{\char`\}}
\def\PYZca{\char`\^}
\def\PYZam{\char`\&}
\def\PYZlt{\char`\<}
\def\PYZgt{\char`\>}
\def\PYZsh{\char`\#}
\def\PYZpc{\char`\%}
\def\PYZdl{\char`\$}
\def\PYZhy{\char`\-}
\def\PYZsq{\char`\'}
\def\PYZdq{\char`\"}
\def\PYZti{\char`\~}
% for compatibility with earlier versions
\def\PYZat{@}
\def\PYZlb{[}
\def\PYZrb{]}
\makeatother


    % Exact colors from NB
    \definecolor{incolor}{rgb}{0.0, 0.0, 0.5}
    \definecolor{outcolor}{rgb}{0.545, 0.0, 0.0}



    
    % Prevent overflowing lines due to hard-to-break entities
    \sloppy 
    % Setup hyperref package
    \hypersetup{
      breaklinks=true,  % so long urls are correctly broken across lines
      colorlinks=true,
      urlcolor=urlcolor,
      linkcolor=linkcolor,
      citecolor=citecolor,
      }
    % Slightly bigger margins than the latex defaults
    
    \geometry{verbose,tmargin=1in,bmargin=1in,lmargin=1in,rmargin=1in}
    
    

    \begin{document}
    
    
    \maketitle
    
    

    
    \begin{Verbatim}[commandchars=\\\{\}]
{\color{incolor}In [{\color{incolor}1}]:} \PY{k+kn}{import} \PY{n+nn}{pandas} \PY{k}{as} \PY{n+nn}{pd}
        \PY{k+kn}{import} \PY{n+nn}{numpy} \PY{k}{as} \PY{n+nn}{np}
        \PY{k+kn}{import} \PY{n+nn}{qexpy} \PY{k}{as} \PY{n+nn}{q}
        \PY{n}{q}\PY{o}{.}\PY{n}{set\PYZus{}print\PYZus{}style}\PY{p}{(}\PY{l+s+s2}{\PYZdq{}}\PY{l+s+s2}{latex}\PY{l+s+s2}{\PYZdq{}}\PY{p}{)}
\end{Verbatim}

    
    
    
    
    \section{Lab 8}\label{lab-8}

\subsubsection{Viraj Bangari}\label{viraj-bangari}

\paragraph{2017-03-17}\label{section}

\begin{itemize}
\tightlist
\item
  Set up the following apparatus:
\end{itemize}

 Figure 1: Diagram of Apparatus

 Figure 2: Detailed Wiring Diagram of Diagram Beam

\begin{itemize}
\tightlist
\item
  Use Leybold power supply for adjusting filament voltge and grid bias.
\item
  Use BKPrecision DC Power supply 1630.
\item
  Use two BK Precision 2831D multimeters for measuring filament voltage
  and filment current. Accuracy is +/- 1 last digit
\item
  Using the Leybold power supply, turn both knobs to zero. Turn it on
\item
  Set anode voltge to 150V. Increase current in Helmholtz until electron
  orbit diameter is 8 cm. Rotate tubes in wood clamps so that the beams
  forms a closed circle.
\item
  Increase the anode voltage to 250V in 20V increasments. Measure the
  current requiered to mainting electrons in circular orbit with 8 cm.
\end{itemize}

    \begin{Verbatim}[commandchars=\\\{\}]
{\color{incolor}In [{\color{incolor}2}]:} \PY{n+nb}{print}\PY{p}{(}\PY{l+s+s2}{\PYZdq{}}\PY{l+s+s2}{Table 1: Voltage and Current for D = 8.0 +/\PYZhy{} 0.2 cm (CCW Beam)}\PY{l+s+s2}{\PYZdq{}}\PY{p}{)}
        \PY{n}{pd}\PY{o}{.}\PY{n}{read\PYZus{}excel}\PY{p}{(}\PY{l+s+s1}{\PYZsq{}}\PY{l+s+s1}{./data/Data.xlsx}\PY{l+s+s1}{\PYZsq{}}\PY{p}{,} \PY{l+m+mi}{0}\PY{p}{)}
\end{Verbatim}

    \begin{Verbatim}[commandchars=\\\{\}]
Table 1: Voltage and Current for D = 8.0 +/- 0.2 cm (CCW Beam)

    \end{Verbatim}

            \begin{Verbatim}[commandchars=\\\{\}]
{\color{outcolor}Out[{\color{outcolor}2}]:}    Anode Voltage [V] +/- 0.1 [V]  Filament Current [A] +/- 0.001[A]
        0                          149.5                              1.226
        1                          169.9                              1.332
        2                          190.3                              1.360
        3                          210.0                              1.451
        4                          229.9                              1.517
        5                          250.0                              1.597
\end{Verbatim}
        
    \begin{itemize}
\tightlist
\item
  Turn tube 180 degrees, and reverse the direction of the current,
  reverse the polarization.
\end{itemize}

    \begin{Verbatim}[commandchars=\\\{\}]
{\color{incolor}In [{\color{incolor}3}]:} \PY{n+nb}{print}\PY{p}{(}\PY{l+s+s2}{\PYZdq{}}\PY{l+s+s2}{Table 2: Voltage and Current for D = 8.0 +/\PYZhy{} 0.2 cm with reversed current polarization (CW Beam)}\PY{l+s+s2}{\PYZdq{}}\PY{p}{)}
        \PY{n}{pd}\PY{o}{.}\PY{n}{read\PYZus{}excel}\PY{p}{(}\PY{l+s+s1}{\PYZsq{}}\PY{l+s+s1}{./data/Data.xlsx}\PY{l+s+s1}{\PYZsq{}}\PY{p}{,} \PY{l+m+mi}{1}\PY{p}{)}
\end{Verbatim}

    \begin{Verbatim}[commandchars=\\\{\}]
Table 2: Voltage and Current for D = 8.0 +/- 0.2 cm with reversed current polarization (CW Beam)

    \end{Verbatim}

            \begin{Verbatim}[commandchars=\\\{\}]
{\color{outcolor}Out[{\color{outcolor}3}]:}    Anode Voltage [V] +/- 0.1 [V]  Filament Current [A] +/- 0.001[A]
        0                          150.3                              1.408
        1                          169.7                              1.472
        2                          189.7                              1.551
        3                          209.8                              1.683
        4                          230.2                              1.645
        5                          250.3                              1.720
\end{Verbatim}
        
    \begin{itemize}
\tightlist
\item
  Repeat steps above, except using D = 10cm
\end{itemize}

    \begin{Verbatim}[commandchars=\\\{\}]
{\color{incolor}In [{\color{incolor}4}]:} \PY{n+nb}{print}\PY{p}{(}\PY{l+s+s2}{\PYZdq{}}\PY{l+s+s2}{Table 3: Voltage and Current for D = 10.0 +/\PYZhy{} 0.2 cm (CCW Beam)}\PY{l+s+s2}{\PYZdq{}}\PY{p}{)}
        \PY{n}{pd}\PY{o}{.}\PY{n}{read\PYZus{}excel}\PY{p}{(}\PY{l+s+s1}{\PYZsq{}}\PY{l+s+s1}{./data/Data.xlsx}\PY{l+s+s1}{\PYZsq{}}\PY{p}{,} \PY{l+m+mi}{2}\PY{p}{)}
\end{Verbatim}

    \begin{Verbatim}[commandchars=\\\{\}]
Table 3: Voltage and Current for D = 10.0 +/- 0.2 cm (CCW Beam)

    \end{Verbatim}

            \begin{Verbatim}[commandchars=\\\{\}]
{\color{outcolor}Out[{\color{outcolor}4}]:}    Anode Voltage [V] +/- 0.1 [V]  Filament Current [A] +/- 0.001[A]
        0                          150.0                              0.984
        1                          169.9                              1.018
        2                          190.0                              1.092
        3                          210.5                              1.146
        4                          230.5                              1.200
        5                          249.6                              1.279
\end{Verbatim}
        
    \begin{Verbatim}[commandchars=\\\{\}]
{\color{incolor}In [{\color{incolor}5}]:} \PY{n+nb}{print}\PY{p}{(}\PY{l+s+s2}{\PYZdq{}}\PY{l+s+s2}{Table 4: Voltage and Current for D = 10.0 +/\PYZhy{} 0.2 cm with reversed current polarization (CW Beam)}\PY{l+s+s2}{\PYZdq{}}\PY{p}{)}
        \PY{n}{pd}\PY{o}{.}\PY{n}{read\PYZus{}excel}\PY{p}{(}\PY{l+s+s1}{\PYZsq{}}\PY{l+s+s1}{./data/Data.xlsx}\PY{l+s+s1}{\PYZsq{}}\PY{p}{,} \PY{l+m+mi}{3}\PY{p}{)}
\end{Verbatim}

    \begin{Verbatim}[commandchars=\\\{\}]
Table 4: Voltage and Current for D = 10.0 +/- 0.2 cm with reversed current polarization (CW Beam)

    \end{Verbatim}

            \begin{Verbatim}[commandchars=\\\{\}]
{\color{outcolor}Out[{\color{outcolor}5}]:}    Anode Voltage [V] +/- 0.1 [V]  Filament Current [A] +/- 0.001[A]
        0                          150.1                              1.093
        1                          170.0                              1.160
        2                          190.0                              1.227
        3                          209.9                              1.298
        4                          230.1                              1.338
        5                          249.8                              1.394
\end{Verbatim}
        
    \begin{Verbatim}[commandchars=\\\{\}]
{\color{incolor}In [{\color{incolor}6}]:} \PY{c+c1}{\PYZsh{} r = q.Measurement(8, 0.2)/100/2 \PYZsh{} m}
        \PY{c+c1}{\PYZsh{} R =  q.Measurement(15.4, 0.5)/100 \PYZsh{} m}
        \PY{c+c1}{\PYZsh{} print(\PYZdq{}r/R is \PYZob{}\PYZcb{}\PYZdq{}.format(r/R))}
        
        \PY{c+c1}{\PYZsh{} B\PYZus{}Bo = 0.99621}
        \PY{c+c1}{\PYZsh{} print(\PYZdq{}B/Bo is \PYZob{}\PYZcb{}\PYZdq{}.format(B\PYZus{}Bo))}
        
        \PY{c+c1}{\PYZsh{} K = q.Measurement(7.74, 0.04)*1e\PYZhy{}4 \PYZsh{} T/A}
        \PY{c+c1}{\PYZsh{} Kr = B\PYZus{}Bo * K}
        \PY{c+c1}{\PYZsh{} print(\PYZdq{}Kr is \PYZob{}\PYZcb{} (T/A)\PYZbs{}n\PYZdq{}.format(Kr))}
        
        \PY{c+c1}{\PYZsh{} data1 = pd.read\PYZus{}excel(\PYZsq{}./data/Data.xlsx\PYZsq{}, 0).as\PYZus{}matrix() \PYZsh{} CCW}
        \PY{c+c1}{\PYZsh{} data2 = pd.read\PYZus{}excel(\PYZsq{}./data/Data.xlsx\PYZsq{}, 1).as\PYZus{}matrix() \PYZsh{} CW}
        \PY{c+c1}{\PYZsh{} V = 0.5 *(q.Measurement(data1[:, 0][\PYZhy{}1], 0.1) + q.Measurement(data2[:, 0][\PYZhy{}1], 0.1)) \PYZsh{} V}
        \PY{c+c1}{\PYZsh{} Is = q.Measurement(data1[:, 1][\PYZhy{}1], 0.001) \PYZsh{} A}
        \PY{c+c1}{\PYZsh{} Il = q.Measurement(data2[:, 1][\PYZhy{}1], 0.001) \PYZsh{} A}
        
        \PY{c+c1}{\PYZsh{} print(\PYZdq{}These measurements are for Il = \PYZob{}\PYZcb{} A, Is = \PYZob{}\PYZcb{} A, V = \PYZob{}\PYZcb{} V\PYZdq{}.format(Il, Is, V))}
        
        \PY{c+c1}{\PYZsh{} Be = Kr/2*(Is \PYZhy{} Il)}
        \PY{c+c1}{\PYZsh{} print(\PYZdq{}Be is \PYZob{}\PYZcb{} T\PYZdq{}.format(Be))}
        
        \PY{c+c1}{\PYZsh{} Bt = Kr/2*(Is + Il)}
        \PY{c+c1}{\PYZsh{} print(\PYZdq{}Bt is \PYZob{}\PYZcb{} T\PYZdq{}.format(Bt))}
        
        \PY{c+c1}{\PYZsh{} e\PYZus{}m = 2*V/(Bt**2 * r**2)}
        \PY{c+c1}{\PYZsh{} print(\PYZdq{}e/m is \PYZob{}\PYZcb{} C/kg\PYZdq{}.format(e\PYZus{}m))}
\end{Verbatim}

    \begin{Verbatim}[commandchars=\\\{\}]
{\color{incolor}In [{\color{incolor}7}]:} \PY{c+c1}{\PYZsh{} Location of original data}
        \PY{n}{data\PYZus{}8cm\PYZus{}normal} \PY{o}{=} \PY{n}{pd}\PY{o}{.}\PY{n}{read\PYZus{}excel}\PY{p}{(}\PY{l+s+s1}{\PYZsq{}}\PY{l+s+s1}{./data/Data.xlsx}\PY{l+s+s1}{\PYZsq{}}\PY{p}{,} \PY{l+m+mi}{0}\PY{p}{)}\PY{o}{.}\PY{n}{as\PYZus{}matrix}\PY{p}{(}\PY{p}{)}
        \PY{n}{data\PYZus{}8cm\PYZus{}reverse} \PY{o}{=} \PY{n}{pd}\PY{o}{.}\PY{n}{read\PYZus{}excel}\PY{p}{(}\PY{l+s+s1}{\PYZsq{}}\PY{l+s+s1}{./data/Data.xlsx}\PY{l+s+s1}{\PYZsq{}}\PY{p}{,} \PY{l+m+mi}{1}\PY{p}{)}\PY{o}{.}\PY{n}{as\PYZus{}matrix}\PY{p}{(}\PY{p}{)}
        \PY{n}{data\PYZus{}10cm\PYZus{}normal} \PY{o}{=} \PY{n}{pd}\PY{o}{.}\PY{n}{read\PYZus{}excel}\PY{p}{(}\PY{l+s+s1}{\PYZsq{}}\PY{l+s+s1}{./data/Data.xlsx}\PY{l+s+s1}{\PYZsq{}}\PY{p}{,} \PY{l+m+mi}{2}\PY{p}{)}\PY{o}{.}\PY{n}{as\PYZus{}matrix}\PY{p}{(}\PY{p}{)}
        \PY{n}{data\PYZus{}10cm\PYZus{}reverse} \PY{o}{=} \PY{n}{pd}\PY{o}{.}\PY{n}{read\PYZus{}excel}\PY{p}{(}\PY{l+s+s1}{\PYZsq{}}\PY{l+s+s1}{./data/Data.xlsx}\PY{l+s+s1}{\PYZsq{}}\PY{p}{,} \PY{l+m+mi}{3}\PY{p}{)}\PY{o}{.}\PY{n}{as\PYZus{}matrix}\PY{p}{(}\PY{p}{)}
        
        \PY{n}{all\PYZus{}data} \PY{o}{=} \PY{p}{[}\PY{n}{data\PYZus{}8cm\PYZus{}normal}\PY{p}{,} \PY{n}{data\PYZus{}8cm\PYZus{}reverse}\PY{p}{,} \PY{n}{data\PYZus{}10cm\PYZus{}normal}\PY{p}{,} \PY{n}{data\PYZus{}10cm\PYZus{}reverse}\PY{p}{]}
        
        \PY{c+c1}{\PYZsh{} Normalize Voltage}
        \PY{n}{voltages} \PY{o}{=} \PY{p}{[}\PY{p}{]}
        \PY{n}{currents} \PY{o}{=} \PY{p}{[}\PY{p}{]}
        \PY{k}{for} \PY{n}{dataset} \PY{o+ow}{in} \PY{n}{all\PYZus{}data}\PY{p}{:}
            \PY{n}{voltage} \PY{o}{=} \PY{n}{dataset}\PY{p}{[}\PY{p}{:}\PY{p}{,} \PY{l+m+mi}{0}\PY{p}{]}
            \PY{n}{voltage} \PY{o}{\PYZhy{}}\PY{o}{=} \PY{l+m+mf}{0.01} \PY{o}{*} \PY{n}{voltage}
            \PY{n}{voltages}\PY{o}{.}\PY{n}{append}\PY{p}{(}\PY{n}{q}\PY{o}{.}\PY{n}{MeasurementArray}\PY{p}{(}\PY{n}{voltage}\PY{p}{,} \PY{l+m+mf}{0.1}\PY{o}{*}\PY{n}{np}\PY{o}{.}\PY{n}{ones}\PY{p}{(}\PY{n+nb}{len}\PY{p}{(}\PY{n}{voltage}\PY{p}{)}\PY{p}{)}\PY{p}{)}\PY{p}{)}
        
            \PY{n}{current} \PY{o}{=} \PY{n}{dataset}\PY{p}{[}\PY{p}{:}\PY{p}{,} \PY{l+m+mi}{1}\PY{p}{]}
            \PY{n}{currents}\PY{o}{.}\PY{n}{append}\PY{p}{(}\PY{n}{q}\PY{o}{.}\PY{n}{MeasurementArray}\PY{p}{(}\PY{n}{current}\PY{p}{,} \PY{l+m+mf}{0.001}\PY{o}{*}\PY{n}{np}\PY{o}{.}\PY{n}{ones}\PY{p}{(}\PY{n+nb}{len}\PY{p}{(}\PY{n}{current}\PY{p}{)}\PY{p}{)}\PY{p}{)}\PY{p}{)}
\end{Verbatim}

    \begin{Verbatim}[commandchars=\\\{\}]
{\color{incolor}In [{\color{incolor}8}]:} \PY{c+c1}{\PYZsh{} Calculating Kr}
        \PY{n}{K} \PY{o}{=} \PY{n}{q}\PY{o}{.}\PY{n}{Measurement}\PY{p}{(}\PY{l+m+mf}{7.74}\PY{p}{,} \PY{l+m+mf}{0.04}\PY{p}{)}\PY{o}{*}\PY{l+m+mi}{1}\PY{n}{e}\PY{o}{\PYZhy{}}\PY{l+m+mi}{4} \PY{c+c1}{\PYZsh{} T/A}
        \PY{n}{R} \PY{o}{=} \PY{n}{q}\PY{o}{.}\PY{n}{Measurement}\PY{p}{(}\PY{l+m+mf}{15.4}\PY{p}{,} \PY{l+m+mf}{0.5}\PY{p}{)}
        \PY{n}{r4cm} \PY{o}{=} \PY{n}{q}\PY{o}{.}\PY{n}{Measurement}\PY{p}{(}\PY{l+m+mi}{8}\PY{p}{,} \PY{l+m+mf}{0.2}\PY{p}{)}\PY{o}{/}\PY{l+m+mi}{2}
        \PY{n}{r5cm} \PY{o}{=} \PY{n}{q}\PY{o}{.}\PY{n}{Measurement}\PY{p}{(}\PY{l+m+mi}{10}\PY{p}{,} \PY{l+m+mf}{0.2}\PY{p}{)}\PY{o}{/}\PY{l+m+mi}{2}
        
        \PY{n+nb}{print}\PY{p}{(}\PY{l+s+s2}{\PYZdq{}}\PY{l+s+s2}{K}\PY{l+s+s2}{\PYZdq{}}\PY{p}{,} \PY{n}{K}\PY{p}{)}
        \PY{n+nb}{print}\PY{p}{(}\PY{l+s+s2}{\PYZdq{}}\PY{l+s+s2}{R}\PY{l+s+s2}{\PYZdq{}}\PY{p}{,} \PY{n}{R}\PY{p}{)}
        
        \PY{n+nb}{print}\PY{p}{(}\PY{l+s+s2}{\PYZdq{}}\PY{l+s+s2}{r4cm/R}\PY{l+s+s2}{\PYZdq{}}\PY{p}{,} \PY{n}{r4cm}\PY{o}{/}\PY{n}{R}\PY{p}{)}
        \PY{n}{B\PYZus{}Bo4cm} \PY{o}{=} \PY{l+m+mf}{0.99621}
        
        \PY{n+nb}{print}\PY{p}{(}\PY{l+s+s2}{\PYZdq{}}\PY{l+s+s2}{r5cm/R}\PY{l+s+s2}{\PYZdq{}}\PY{p}{,} \PY{n}{r5cm}\PY{o}{/}\PY{n}{R}\PY{p}{)}
        \PY{n}{B\PYZus{}Bo5cm} \PY{o}{=} \PY{l+m+mf}{0.99621}
        
        \PY{n}{Kr4cm} \PY{o}{=} \PY{n}{B\PYZus{}Bo4cm} \PY{o}{*} \PY{n}{K}
        \PY{n}{Kr5cm} \PY{o}{=} \PY{n}{B\PYZus{}Bo5cm} \PY{o}{*} \PY{n}{K}
        
        \PY{n+nb}{print}\PY{p}{(}\PY{l+s+s2}{\PYZdq{}}\PY{l+s+s2}{Kr 4cm}\PY{l+s+s2}{\PYZdq{}}\PY{p}{,} \PY{n}{Kr4cm}\PY{p}{)}
        \PY{n+nb}{print}\PY{p}{(}\PY{l+s+s2}{\PYZdq{}}\PY{l+s+s2}{Kr 5cm}\PY{l+s+s2}{\PYZdq{}}\PY{p}{,} \PY{n}{Kr5cm}\PY{p}{)}
\end{Verbatim}

    \begin{Verbatim}[commandchars=\\\{\}]
K (774 \textbackslash{}pm 4)*10\^{}\{-6\}
R (154 \textbackslash{}pm 5)*10\^{}\{-1\}
r4cm/R (26 \textbackslash{}pm 1)*10\^{}\{-2\}
r5cm/R (32 \textbackslash{}pm 1)*10\^{}\{-2\}
Kr 4cm (771 \textbackslash{}pm 4)*10\^{}\{-6\}
Kr 5cm (771 \textbackslash{}pm 4)*10\^{}\{-6\}

    \end{Verbatim}

    \begin{Verbatim}[commandchars=\\\{\}]
{\color{incolor}In [{\color{incolor}9}]:} \PY{c+c1}{\PYZsh{} Calculating Bt and Be}
        
        \PY{n}{Be4cm} \PY{o}{=} \PY{n}{Kr4cm}\PY{o}{/}\PY{l+m+mi}{2} \PY{o}{*} \PY{p}{(}\PY{n}{currents}\PY{p}{[}\PY{l+m+mi}{0}\PY{p}{]} \PY{o}{\PYZhy{}} \PY{n}{currents}\PY{p}{[}\PY{l+m+mi}{1}\PY{p}{]}\PY{p}{)}
        \PY{n+nb}{print}\PY{p}{(}\PY{l+s+s2}{\PYZdq{}}\PY{l+s+s2}{Be4cm}\PY{l+s+s2}{\PYZdq{}}\PY{p}{,} \PY{n}{Be4cm}\PY{o}{.}\PY{n}{get\PYZus{}error\PYZus{}weighted\PYZus{}mean}\PY{p}{(}\PY{p}{)}\PY{p}{)}
        \PY{n}{Be5cm} \PY{o}{=} \PY{n}{Kr5cm}\PY{o}{/}\PY{l+m+mi}{2} \PY{o}{*} \PY{p}{(}\PY{n}{currents}\PY{p}{[}\PY{o}{\PYZhy{}}\PY{l+m+mi}{2}\PY{p}{]} \PY{o}{\PYZhy{}} \PY{n}{currents}\PY{p}{[}\PY{o}{\PYZhy{}}\PY{l+m+mi}{1}\PY{p}{]}\PY{p}{)}
        \PY{n+nb}{print}\PY{p}{(}\PY{l+s+s2}{\PYZdq{}}\PY{l+s+s2}{Be5cm}\PY{l+s+s2}{\PYZdq{}}\PY{p}{,} \PY{n}{Be5cm}\PY{o}{.}\PY{n}{get\PYZus{}error\PYZus{}weighted\PYZus{}mean}\PY{p}{(}\PY{p}{)}\PY{p}{)}
        
        \PY{n}{Bt4cm} \PY{o}{=} \PY{n}{Kr4cm}\PY{o}{/}\PY{l+m+mi}{2} \PY{o}{*} \PY{p}{(}\PY{n}{currents}\PY{p}{[}\PY{l+m+mi}{0}\PY{p}{]} \PY{o}{+} \PY{n}{currents}\PY{p}{[}\PY{l+m+mi}{1}\PY{p}{]}\PY{p}{)}
        \PY{n+nb}{print}\PY{p}{(}\PY{l+s+s2}{\PYZdq{}}\PY{l+s+s2}{Bt4cm}\PY{l+s+s2}{\PYZdq{}}\PY{p}{,} \PY{n}{Bt4cm}\PY{o}{.}\PY{n}{get\PYZus{}error\PYZus{}weighted\PYZus{}mean}\PY{p}{(}\PY{p}{)}\PY{p}{)}
        \PY{n}{Bt5cm} \PY{o}{=} \PY{n}{Kr5cm}\PY{o}{/}\PY{l+m+mi}{2} \PY{o}{*} \PY{p}{(}\PY{n}{currents}\PY{p}{[}\PY{o}{\PYZhy{}}\PY{l+m+mi}{2}\PY{p}{]} \PY{o}{+} \PY{n}{currents}\PY{p}{[}\PY{o}{\PYZhy{}}\PY{l+m+mi}{1}\PY{p}{]}\PY{p}{)}
        \PY{n+nb}{print}\PY{p}{(}\PY{l+s+s2}{\PYZdq{}}\PY{l+s+s2}{Bt5cm}\PY{l+s+s2}{\PYZdq{}}\PY{p}{,} \PY{n}{Bt5cm}\PY{o}{.}\PY{n}{get\PYZus{}error\PYZus{}weighted\PYZus{}mean}\PY{p}{(}\PY{p}{)}\PY{p}{)}
        
        \PY{n}{Be} \PY{o}{=} \PY{p}{(}\PY{n}{Be4cm}\PY{o}{.}\PY{n}{get\PYZus{}error\PYZus{}weighted\PYZus{}mean}\PY{p}{(}\PY{p}{)} \PY{o}{+} \PY{n}{Be5cm}\PY{o}{.}\PY{n}{get\PYZus{}error\PYZus{}weighted\PYZus{}mean}\PY{p}{(}\PY{p}{)}\PY{p}{)}\PY{o}{/}\PY{l+m+mi}{2}
        \PY{n+nb}{print}\PY{p}{(}\PY{l+s+s2}{\PYZdq{}}\PY{l+s+s2}{Be}\PY{l+s+s2}{\PYZdq{}}\PY{p}{,} \PY{n}{Be}\PY{p}{)}
\end{Verbatim}

    \begin{Verbatim}[commandchars=\\\{\}]
Be4cm (-621 \textbackslash{}pm 3)*10\^{}\{-7\}
Be5cm (-506 \textbackslash{}pm 2)*10\^{}\{-7\}
Bt4cm (1140 \textbackslash{}pm 2)*10\^{}\{-6\}
Bt5cm (901 \textbackslash{}pm 2)*10\^{}\{-6\}
Be (-564 \textbackslash{}pm 2)*10\^{}\{-7\}

    \end{Verbatim}

    \begin{Verbatim}[commandchars=\\\{\}]
{\color{incolor}In [{\color{incolor}10}]:} \PY{c+c1}{\PYZsh{} em for 4cm}
         \PY{n}{V4cm} \PY{o}{=} \PY{p}{(}\PY{n}{voltages}\PY{p}{[}\PY{l+m+mi}{0}\PY{p}{]} \PY{o}{+} \PY{n}{voltage}\PY{p}{[}\PY{l+m+mi}{1}\PY{p}{]}\PY{p}{)}\PY{o}{/}\PY{l+m+mi}{2}
         \PY{n}{e\PYZus{}m4cm} \PY{o}{=} \PY{l+m+mi}{2}\PY{o}{*}\PY{n}{V4cm} \PY{o}{/} \PY{p}{(}\PY{n}{Bt4cm} \PY{o}{*} \PY{n}{r4cm}\PY{o}{/}\PY{l+m+mi}{100}\PY{p}{)}\PY{o}{*}\PY{o}{*}\PY{l+m+mi}{2}
         
         \PY{n+nb}{print}\PY{p}{(}\PY{n}{e\PYZus{}m4cm}\PY{o}{.}\PY{n}{get\PYZus{}error\PYZus{}weighted\PYZus{}mean}\PY{p}{(}\PY{p}{)}\PY{p}{)}
\end{Verbatim}

    \begin{Verbatim}[commandchars=\\\{\}]
(171 \textbackslash{}pm 4)*10\^{}\{9\}

    \end{Verbatim}

    \begin{Verbatim}[commandchars=\\\{\}]
{\color{incolor}In [{\color{incolor}11}]:} \PY{c+c1}{\PYZsh{} em for 8cm}
         \PY{n}{V5cm} \PY{o}{=} \PY{p}{(}\PY{n}{voltages}\PY{p}{[}\PY{o}{\PYZhy{}}\PY{l+m+mi}{1}\PY{p}{]} \PY{o}{+} \PY{n}{voltages}\PY{p}{[}\PY{o}{\PYZhy{}}\PY{l+m+mi}{2}\PY{p}{]}\PY{p}{)}\PY{o}{/}\PY{l+m+mi}{2}
         \PY{n}{e\PYZus{}m5cm} \PY{o}{=} \PY{l+m+mi}{2}\PY{o}{*}\PY{n}{V5cm} \PY{o}{/} \PY{p}{(}\PY{n}{Bt5cm} \PY{o}{*} \PY{n}{r5cm}\PY{o}{/}\PY{l+m+mi}{100}\PY{p}{)}\PY{o}{*}\PY{o}{*}\PY{l+m+mi}{2}
         
         \PY{n+nb}{print}\PY{p}{(}\PY{n}{e\PYZus{}m5cm}\PY{o}{.}\PY{n}{get\PYZus{}error\PYZus{}weighted\PYZus{}mean}\PY{p}{(}\PY{p}{)}\PY{p}{)}
\end{Verbatim}

    \begin{Verbatim}[commandchars=\\\{\}]
(188 \textbackslash{}pm 3)*10\^{}\{9\}

    \end{Verbatim}

    \begin{Verbatim}[commandchars=\\\{\}]
{\color{incolor}In [{\color{incolor}12}]:} \PY{n}{e\PYZus{}m} \PY{o}{=} \PY{p}{(}\PY{n}{e\PYZus{}m4cm}\PY{o}{.}\PY{n}{get\PYZus{}error\PYZus{}weighted\PYZus{}mean}\PY{p}{(}\PY{p}{)} \PY{o}{+} \PY{n}{e\PYZus{}m5cm}\PY{o}{.}\PY{n}{get\PYZus{}error\PYZus{}weighted\PYZus{}mean}\PY{p}{(}\PY{p}{)}\PY{p}{)}\PY{o}{/}\PY{l+m+mi}{2}
         \PY{n+nb}{print}\PY{p}{(}\PY{n}{e\PYZus{}m}\PY{p}{)}
\end{Verbatim}

    \begin{Verbatim}[commandchars=\\\{\}]
(180 \textbackslash{}pm 2)*10\^{}\{9\}

    \end{Verbatim}


    % Add a bibliography block to the postdoc
    
    
    
    \end{document}
